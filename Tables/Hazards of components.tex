

\begin{center}
\begin{longtable}[c]{|p{2.3cm}|p{3.2cm}|p{2.2cm}|p{7cm}|p{5cm}|p{4.5cm}|}

\caption{Components' potential health and safety risks}
 \hline \rowcolor[HTML]{CBCEFB} 
\textbf{Component} & \textbf{Aquatic toxicity} & \textbf{Global warning potential [kg CO$_2$-eq/kg]} & \textbf{Inhalation effects} & \textbf{Contact effects} & \textbf{Flammability}  \hline

\hline
CaCl$_2$ &
  
Should not be released into the environment &
0.89 &
May cause severe irritation of the upper respiratory tract with pain, burns, and inflammation &
Contact with eyes may cause severe irritation, and possible eye burns. Contact with skin causes irritation and possible burns, especially if the skin is wet or moist &
Nonflammable substance. Use agent most appropriate to extinguish surrounding fire.
\hline
CaCO$_3$ &
Should not be released into the environment &
0.89 &
Inhaling calcium carbonate can irritate the nose, throat and lungs causing coughing &
Contact can irritate the skin and eyes &
Is not combustible. When heated to decomposition it emits an acid smoke and irritating vapors. Use dry chemical, CO$_2$, water spray or foam as extinguishing agents.
\hline
\hline


CaO &
Avoid release to the environment &
0.15 &
Breathing CaO, can irritate the lungs causing coughing  and shortness of breath. Higher exposures may cause a pulmonary edema. Long-term exposure can irritate the nose causing a hole in the bone dividing the inner nose and can cause brittle nails and thickening of the cracking of the skin. &
CaO can severely irritate and burn the skin and eyes with possible eye damage. & 
Substance is nonflammable and non-combustible but is water reactive. Water must not be used as extinguishing media for safety reasons
\hline
\hline

Ca(OH)$_2$ &
Should not be released into the environment &
Not specified &
Breathing Ca(OH)$_2$  can irritate the nose, throat and lungs causing coughing , wheezing and/or shortness of breath &
Contact can severely irritate and burn the skin and eyes with possible eye damage & 
Negligible fire hazard. Use water spray, dry chemical, carbon dioxide, or chemical foam for extinguishing purposes.
\hline

CO$_2$ &
Prevent further leakage or spillage if safe to do so &
1 &
Low to medium concentrations can affect blood circulation, affect acidity of body fluids and respiratory difficulties. At high concentrations, can cause unconsciousness/death. & 
No effects &
Contains gas under pressure; may explode if heated. 
\hline

Na$_2$CO$_3$ &
Prevent from reaching drains, sewers or waterway. Collect contaminated soil for characterization &
0.59 &
None &
Causes serious eye irritation &
Negligible fire hazard.May be combustible at high temperatures.
Use water spray, dry chemical, carbon dioxide, or chemical foam for extinguishing purposes. 
\hline

NaHCO$_3$ &
Should not be released into the environment &
1.17 &
May cause respiratory tract irritation. May be harmful if inhaled. &
Causes eye irritation. Causes skin irritation. May be harmful if absorbed through the skin. May be harmful if swallowed. Causes gastrointestinal tract irritation. &
Non-combustible, substance itself does not burn but may decompose upon heating to produce corrosive and/or toxic fumes Use water spray, dry chemical, carbon dioxide, or chemical foam. 
\hline 
\hline 
NaCl &
Should not be released into the environment &
0.20 &
May cause respiratory tract irritation. May be harmful if inhaled.&
May cause eye irritation. Exposure to solid may cause pain and redness. May cause skin irritation. May be harmful if absorbed through the skin. May cause irritation of the digestive tract. May be harmful if swallowed. Ingestion of large amounts may cause nausea and vomiting, rigidity, or convulsions. Continued exposure can produce coma, dehydration, and internal organ failure. &
Negligible fire hazard.  May be combustible at high temperatures.
 Use water spray, dry chemical, carbon dioxide, or chemical foam.
 \hline 
 
CH$_4$ &
Prevent further leakage or spillage if safe to do so.&
28 &
None &
None &
Extremely flammable gas. Contains gas under pressure; may explode if heated. Suitable extinguishing media: Water, dry powder, foam. 
\hline
\end{longtable}
\end{center}