\section{Risk Matrix}



\begin{table}[H]
\centering
\caption{Likelihood categories defined for the severity table}
\begin{tabular}{|l|l|l|l|}
\hline
\textbf{L} & \textbf{Likelihood} & \textbf{Probability}                        & \textbf{Frequency} \\ \hline
A          & Improbable          & p$_{once}$   \textless 0.001                  & 1 every 1000 years \\ \hline
B          & Unlikely            & 0.001   \textless{}   p$_{once}$ \textless 0.01 & 1 every 100 years  \\ \hline
C          & Possible            & 0.01   \textless{}   p$_{once}$ \textless 0.1   & 1 every 10 years   \\ \hline
D          & Probable            & 0.1   \textless{}   p$_{once}$ \textless 1      & 1 per year         \\ \hline
E          & Frequent            & p$_{several}$   = 1                           & Several per year   \\ \hline
\end{tabular}
\end{table}

\begin{landscape}

\section{Component Hazards to People and the Environment} 



\begin{center}
\begin{longtable}[c]{|p{2.3cm}|p{3.2cm}|p{2.2cm}|p{7cm}|p{5cm}|p{4.5cm}|}

\caption{Components' potential health and safety risks}
 \hline \rowcolor[HTML]{CBCEFB} 
\textbf{Component} & \textbf{Aquatic toxicity} & \textbf{Global warning potential [kg CO$_2$-eq/kg]} & \textbf{Inhalation effects} & \textbf{Contact effects} & \textbf{Flammability}  \hline

\hline
CaCl$_2$ &
  
Should not be released into the environment &
0.89 &
May cause severe irritation of the upper respiratory tract with pain, burns, and inflammation &
Contact with eyes may cause severe irritation, and possible eye burns. Contact with skin causes irritation and possible burns, especially if the skin is wet or moist &
Nonflammable substance. Use agent most appropriate to extinguish surrounding fire.
\hline
CaCO$_3$ &
Should not be released into the environment &
0.89 &
Inhaling calcium carbonate can irritate the nose, throat and lungs causing coughing &
Contact can irritate the skin and eyes &
Is not combustible. When heated to decomposition it emits an acid smoke and irritating vapors. Use dry chemical, CO$_2$, water spray or foam as extinguishing agents.
\hline
\hline


CaO &
Avoid release to the environment &
0.15 &
Breathing CaO, can irritate the lungs causing coughing  and shortness of breath. Higher exposures may cause a pulmonary edema. Long-term exposure can irritate the nose causing a hole in the bone dividing the inner nose and can cause brittle nails and thickening of the cracking of the skin. &
CaO can severely irritate and burn the skin and eyes with possible eye damage. & 
Substance is nonflammable and non-combustible but is water reactive. Water must not be used as extinguishing media for safety reasons
\hline
\hline

Ca(OH)$_2$ &
Should not be released into the environment &
Not specified &
Breathing Ca(OH)$_2$  can irritate the nose, throat and lungs causing coughing , wheezing and/or shortness of breath &
Contact can severely irritate and burn the skin and eyes with possible eye damage & 
Negligible fire hazard. Use water spray, dry chemical, carbon dioxide, or chemical foam for extinguishing purposes.
\hline

CO$_2$ &
Prevent further leakage or spillage if safe to do so &
1 &
Low to medium concentrations can affect blood circulation, affect acidity of body fluids and respiratory difficulties. At high concentrations, can cause unconsciousness/death. & 
No effects &
Contains gas under pressure; may explode if heated. 
\hline

Na$_2$CO$_3$ &
Prevent from reaching drains, sewers or waterway. Collect contaminated soil for characterization &
0.59 &
None &
Causes serious eye irritation &
Negligible fire hazard.May be combustible at high temperatures.
Use water spray, dry chemical, carbon dioxide, or chemical foam for extinguishing purposes. 
\hline

NaHCO$_3$ &
Should not be released into the environment &
1.17 &
May cause respiratory tract irritation. May be harmful if inhaled. &
Causes eye irritation. Causes skin irritation. May be harmful if absorbed through the skin. May be harmful if swallowed. Causes gastrointestinal tract irritation. &
Non-combustible, substance itself does not burn but may decompose upon heating to produce corrosive and/or toxic fumes Use water spray, dry chemical, carbon dioxide, or chemical foam. 
\hline 
\hline 
NaCl &
Should not be released into the environment &
0.20 &
May cause respiratory tract irritation. May be harmful if inhaled.&
May cause eye irritation. Exposure to solid may cause pain and redness. May cause skin irritation. May be harmful if absorbed through the skin. May cause irritation of the digestive tract. May be harmful if swallowed. Ingestion of large amounts may cause nausea and vomiting, rigidity, or convulsions. Continued exposure can produce coma, dehydration, and internal organ failure. &
Negligible fire hazard.  May be combustible at high temperatures.
 Use water spray, dry chemical, carbon dioxide, or chemical foam.
 \hline 
 
CH$_4$ &
Prevent further leakage or spillage if safe to do so.&
28 &
None &
None &
Extremely flammable gas. Contains gas under pressure; may explode if heated. Suitable extinguishing media: Water, dry powder, foam. 
\hline
\end{longtable}
\end{center}

\end{landscape}

\section{12 Principles of Green Chemistry}
\label{section:12principles}
\textbf{1. Prevention:} It is better to prevent waste than to treat or clean up waste after it has been created. The Solvay process avoids a lot of waste due to the recycle of most of its intermediates such as CO$_2$. Additionally, the implementation of a calcium loop as opposed to ammonia results in the production of a safer byproduct (CaCl$_2$ vs NH$_4$Cl in the modified Solvay process). 

\noindent \textbf{2. Atom economy:} Synthetic methods should be designed to maximize the incorporation of all materials used in the process into the final product. In the synthetic method chosen many of the feeds and byproducts are recycled. 

\noindent \textbf{3. Less hazardous chemical synthesis:} Wherever practicable, synthetic methods should be designed to use and generate substances that possess little or no toxicity to human health and the environment. The synthesis method chosen, avoids the use ammonia and introduces a calcium loop. This was done because ammonia is a much more hazardous chemical which has various negative impacts for the environment and health of humans. By avoiding its use and introducing CaO as a replacement, one of the main hazards of the Solvay process has been prevented.

\noindent \textbf{4. Designing safer chemicals:} Chemical products should be designed to preserve efficacy of function while reducing toxicity. The altered Solvay process route chosen ensured the production of soda ash in the safest method. 

\noindent \textbf{5. Safer solvents and auxiliaries:} The use of auxiliary substances (e.g., solvents, separation agents, etc.) should be made unnecessary wherever possible and, innocuous when used. Solvents used were water-based and therefore non-toxic and non-hazardous. 

\noindent \textbf{6. Design for energy efficiency:} Energy requirements should be recognized for their environmental and economic impacts and should be minimized. Synthetic methods should be conducted at ambient temperature and pressure. Heat integration methods will be implemented where hot streams need to be cooled, preferably using colder streams at ambient temperature. No pressurisation is required for the system; all units will operate at 1 atm. 

\noindent \textbf{7. Use of renewable feedstocks: } A raw material or feedstock should be renewable rather than depleting whenever technically and economically practicable. 

\noindent \textbf{8. Reduce derivatives:} Unnecessary derivatisation should be minimized or avoided if possible, because such steps require additional reagents and can generate waste. The Solvay synthesis step was selected to minimize the steps and intermediate reagents. Extra steps involving the purification of the feedstock was avoided by buying the brine instead of energy-intensive techniques to obtain the desired saline content.

\noindent \textbf{9. Catalysis:} Catalytic reagents (as selective as possible) are superior to stoichiometric reagents. This will be the case for the DSR as steam catalyses the first calcination reaction in R-1. 

\noindent \textbf{10. Design for degradation:} Chemical products should be designed so that at the end of their function they break down into innocuous degradation products and do not persist in the environment. Although Na$_2$CO$_3$ is non-toxic, it is non-biodegradable and is designed in that way intentionally for the consumer (in the production of glass and car segments). 

\noindent \textbf{11. Real-time analysis for pollution prevention:} Analytical methodologies need to be further developed to allow for real-time, in-process monitoring and control prior to the formation of hazardous substances. Control measures such as pressure relief valves, temperature sensors, NO$_X$ concentration sensors and flow control valves will be integrated in Solution A’s next stages of the plant design.

\noindent \textbf{12. Inherently safer chemistry for accident prevention:} Substances and the form of a substance used in a chemical process should be chosen to minimize the potential for chemical accidents, including releases, explosions, and fires. The elimination of NH$_3$ in the modified process is paramount in ensuring inherent safety and so accidents and any hazards involving NH$_3$ will be avoided at all costs. 
\vspace{-20pt}
