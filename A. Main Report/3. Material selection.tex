\section{Material Selection for Plant Construction}
\vspace{-10pt}
Material selection for the process units was required so as not to contaminate any process flows and ensure that the right material and grade was selected, depending on the conditions of each unit as well as the flow rates and properties of the species at each stage of the process. Furthermore, careful consideration was made to temper the effects of corrosion and avoid any contamination from the material(s) to the process. In addition to chemical factors, physical factors such as tensile strength, temperature ranges, sterilisation and expected operating lifetime were also regarded as pertinent factors for the choice of materials. Solution A's Solvay plant will use 316L stainless steel as a base for most operation units, where temperatures are below 450\textdegree C and pressures of approximately 1 bar. 
\vspace{-10pt}
\subsection{Chemical Factor}
\vspace{-10pt}
The presence of Ca(OH)$_2$ in parts of the process means that some units must be manufactured to be resistant to alkaline corrosion attack. Although 316L SS is more resistant to chlorides than 304 SS, it is still not deemed as a resistant enough material operate long term, considering the pH of limewater is 12.4 \citep{resistant}. The nickel-molybdenum-chromium alloy, HAYNES 214, was selected for R-1 and R-2.The nickel-based alloy was selected due to its ability to withstand severe conditions such as high temperatures, and corrosion \citep{nickelalloy} \citep{HAYNES}. 


\vspace{-10pt}
\subsection{Physical Factors}
\vspace{-10pt}
\subsubsection{Equipment Strength}
\vspace{-10pt}
There are two compressors within the soda-ash plant (COMP-1 and COMP-2) as shown in the PFD. For these process units it is necessary to consider the strength of the equipment. Thus, cast iron was selected since it has an excellent compressive strength, as well as the ability to withstand loads that would reduce its size. Additionally, it is a perfect shock absorber and is very durable, which makes it the right material to be used in heavy duty environments and high-pressure conditions \citep{cast iron}.
\subsubsection{Thermal Properties}
\vspace{-10pt}
Throughout the whole process, most units don´t reach very high temperatures. Therefore, R-3, R-4, R-5 (reference PFD) and all flash drums will be manufactured from 316L stainless steel. However there are two process units which are the CaCO$_3$ calcinator (R-1) and the carbonator (R-2) that operate under extremely high temperatures, where the default metal is not suitable for operation at these ranges, assuming a plant lifetime of 30 to 40 years. For R-1, a nickel-chromium-aluminium-iron alloy, HAYNES 214 alloy (UNS N07214) was selected, which is specifically designed to withstand temperatures as high as 1260\textdegree C and offers outstanding oxidation resistance \citep{Haynes}. For R-2, HAYNES 242 alloy (UNS N10242) was the preferred material. This is an age-hardenable, nickel-molybdenum-chromium alloy which offers excellent strength to 705\textdegree C, low thermal expansion characteristics, good oxidation resistance up to 815 \textdegree C, excellent low-cycle fatigue properties and very good thermal stability \cite{Haynes}. 
\newline
\noindent For the design of the various coolers and heater, aluminum was selected. This material has the optimum thermal properties and corrosion resistance to make it and ideal choice. The thermal conductivity of pure aluminum is 237 W/($m \times K$) or 160 W/($m\times K$) for most alloys making aluminium the third most thermally conductive material and arguably the most cost-effective. In addition, aluminium offers a specific heat of 0.44 J/($g\times \textdegree C$), making it very nearly as efficient at diffusing heat as copper \citep{cooler}.
\newline
\noindent For the storage of CO$_2$, a cylinder tank was selected where CO$_2$ is kept in liquid phase.The CO$_2$ can be stored at room temperature and at a maximum pressure of 25 bar. Consequently, the material chosen for the inner vessel is carbon steel P355 NL2 for low temperature use and the outer jacket will be insulated \citep{CO2 storage}. 
\vspace{-10pt}
\subsubsection{Sterilisation and Cleaning Properties}
\vspace{-10pt}
For a good manufacturing practice, regular cleaning and maintenance is very important. The main material used throughout the plant which is stainless steel and nickel alloys, are materials with good sterilisation and cleaning properties.
\vspace{-10pt}
\subsection{Storage conditions}
\vspace{-10pt}
A main priority of Solution A industries is maintaining the quality and purity of the soda ash produced. For this reason, special considerations have to be taken into account when storing soda ash. Soda ash has to be stored in a dry place to avoid hydration, crust formation and hardening because if sodium carbonate is stored under moist conditions, its alkalinity decreases due to the absorption of moisture and CO$_2$ from the atmosphere. Additionally, necessary precautions have to be taken to prevent contamination by other  nearby  stored  products, and to prevent the release of soda ash dust during handling. Accordingly, appropriate exhaust ventilation at places where dust is formed has to be incorporated. The material chosen for its storage is XLPE for its longevity since the lifespan of the plant is 35 years, durability, and safety \citep{storage soda}. 
\newline
\noindent The storage conditions of the raw materials has to be taken into account as well. In the case of CaCO$_3$ it needs to be stored in a tightly closed container in a cool, well-ventilated area· Moreover it has to be stored away from acids, aluminum, ammonium salts, hydrogen and magnesium since its not compatible with them and away from fluorine because when they are in contact it ignites \citep{storage CaCO3}. For this material, its storage tank is going to be made out of aluminium because it´s lightweight, corrosion resistant and there is minimal risk of contamination \citep{aluminium}. Finally, for the storage of brine it´s necessary to keep it in a dry, cool place, away from direct sunlight and extremely high or low temperatures. In addition it cannot be in contact with strong acids, strong bases, strong oxidizers and water-reactive materials. The material chosen for the storage of brine is XLPE because it is not prone to damage the brine and it is a cost effective material \citep{storage brine}. 
\subsection{Material Choice for Pipelines}
\vspace{-10pt}
The material choice for the pipelines carrying the flows from R-3, R-4, R-5 (ref PFD),all flash drums, the compressors and the heat exchangers is 316L stainless steel . Since the flow coming out of these units is at almost atmospheric pressure and not very high temperatures, there is no need for special considerations. Contrarily, the flows coming out or R-1 and R-2 are in more extreme conditions (higher temperature and pressure). Consequently, the pipelines are going to be made of the same material as the reactor to withstand the conditions of the flow. For the the pipelines connected to the CO$_2$ storage tank, stainless steel piping was chosen for reliability and long life \citep{CO2 storage}. 




