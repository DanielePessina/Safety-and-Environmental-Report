\section{Considerations in Process Synthesis}
\vspace{-10pt}
\subsection{Relevant laws and Regulations}
\vspace{-10pt}
\subsubsection{Control of Substances Hazardous to Health (COSHH) Regulations, 2002}
\vspace{-10pt}
COSHH is the law that requires employers to control substances that are hazardous to the health of an individual. Employees’ exposure to hazardous substances can be significantly reduced by taking the relevant steps, which include i) identifying the health hazards, ii) conducting a risk assessment iii) providing control measures to reduce harm to health iv) ensuring that all control measures are in optimal working order v) planning for emergency situations. Manufacturing companies must provide Material Safety Data Sheets (MSDS), which cover proper use, storage and handling of chemicals in an appropriate manner. This also includes Workplace Exposure Limits (WEL) as a guide for workers to monitor exposure to hazardous substances in the workplace. 
\vspace{-10pt}
\subsubsection{Dangerous Substances and Explosive Atmosphere Regulations (DSEAR), 2002}
\vspace{-10pt}
Dangerous substances include any substances present in the workplace that have the potential to cause harm to people from the release of an explosion, fire or corrosion of metals. The DSEAR was introduced to ensure the safety of workers from the risks included the hazards formerly stated. 
\vspace{-10pt}
\subsubsection{Occupational Health and Safety Act No. 85, 1993}
\vspace{-10pt}
This legislation imposes occupational health and safety standards for employers and workers in and around the vicinity of the plant. All employers must comply with the relevant exposure limits specified by MSDS, HAZOP and emergency procedures. The Occupational Safety and Health Administration also administers whistleblower statutes.
\vspace{-10pt}
\subsubsection{European Union (EU) Pressure Equipment Directive PED, 2016}
\vspace{-10pt}
The PED applied to the design, manufacture and conformity assessment of stationary pressure vessels with a maximum allowable 0.5 bar \citep{PED} This excludes most transport tanks which covers explosive and toxic raw materials. Pressure vessels and assemblies within the scope of PED must bear a CE mark, and are categorised according to their hazard (from I to V). 
\vspace{-10pt}
\subsubsection{Air Quality Standards Regulations, 2010}
\vspace{-10pt}
Health based standards and objectives for a number of pollutants in the air are set, whereby limits and permitted exceedences are specified for all countries in the EU \citep{air_quality}. The targets are set at national levels and are based on the average exposure indicator, which is calculated on the basis of a three-year period.

\begin{landscape}
\subsection{Past Incidents}
    \begin{center}
\begin{longtable}[c]{|p{3cm}|p{7.5cm}|p{7.5cm}|p{7.5cm}|}

\caption{Past incidents that involved the largest hazard and operation in Solvay plants}
\hline
\rowcolor[HTML]{CBCEFB} 
\textbf{Incident} & \textbf{Description} & \textbf{Consequences} & \textbf{Lessons   Learnt}   \hline

\hline
The Toxic White Beaches of Rosignano \citep{whitebeaches}&
  
The Solvay chemical plant in Rosignano   had been disposing their by-products     (mixture of calcium chloride   and limestone) in the sea.Many toxic   chemicals are mixed in with   the CaCl$_2$ e.g. mercury, arsenic,   cadmium, chromium, lead and ammonia,   which are harmful to humans and   animals. This is because limestone (main  feed in the Solvay process) has impurities  consisting of heavy metals. &
  
Due to the disposal of toxic   chemicals in the beach, the   Spiagge Bianche is among   the 15 most polluted coastal   sites in the Mediterranean Sea.   Between 2008 and 2010,   the town recorded a mortality   rate higher than the regional   average for the same period,   increasing by 2.2 \% for men   and 8.3 \% for women.   The frequency of tumours   and premature mortality (under   age 65) are both above the   regional average by several   percentage points. &
  
The purer the limestone   used in the Solvay  process,   the more suitable the limestone   is for lessening the impact from    the production of soda ash on   the environment. It is also   necessary to make sure that if   by-products are disposed in   the sea, they must  be purified from   toxic chemicals to not contaminate   the waterway.  \hline
TATA Chemicals Incidents \citep{TATA1}, \citep{TATA2} &
Operation of the   plant by   means of unsafe work methods.   Another incident happened on   March 12, 2021, when a large fire     was ignited in the TATA Chemicals   plant which was caused by   malfunctioning of electrical equipment   within an industrial building. A contractor   employee suffered chemical burns when   he was engulfed in hot caustic lime dust.   In addition, a worker fell from a walkway   8 feet high and became trapped to his waist. &
Workers from the TATA   Chemicals factory got   injured due to the lack   of safety measures in   the plant. TATA Chemicals   was fined with   almost \pounds 350,000.&
Both incidents could have   been avoided with regular   assessment of risks and   inspection of work equipment .  
\hline
\hline

Imperial Sugar Company   Dust Explosion and Fire  &
The explosion was fueled   by massive accumulations   of combustible sugar dust   throughout the packaging building. On   February 7, 2008, a huge explosion   and fire occurred at the Imperial Sugar   refinery northwest of Savannah, Georgia. &
This explosion caused 14   deaths and injuring 38 others,   including 14 with serious and   life-threatening burns.&
Combustible dust hazard   awareness should be incorporated   into employee and member   companies’ training programs.   Combustible dust characteristics,   especially ignition energy and   minimum explosible concentration   should also be studied.  In addition,   best practices for minimising   dust accumulation should be   incorporated as well as safe   housekeeping practices.   Finally, specific combustible   dust inspection requirements   should be implemented.  \hline

\end{longtable}
\end{center}
\vspace{-15pt}
Having outlined some of the major incidents involving Solvay plants and hazard that poses the greatest danger (dust), Solution A aims to learn from them to ensure any avoidable crisis are taken care of. Examining past incidents is a good way to focus on minimising risk, and special considerations have taken place to mitigate the hazards associated with soda ash production. 
\vspace{-10pt}
    \subsection{Decision Making}
\vspace{-10pt}
Considerations of the toxic nature of ammonia were taken into account in the evaluation of the safest process route for the production of soda ash. The main aim was to find a feasible way of substituting ammonia (with another base) without compromising the required product specifications. The choice of Na$_2$CO$_3$ synthesis route, raw materials, operating conditions and calcination reactor type were all decided with the safety and environmental considerations in mind. Reactor conditions were also investigated in terms of minimum and maximum temperatures and pressures, in TOPSIS and AHP analysis, as aformentioned in the \textbf{Synthesis Report}. Finally, environmental impact by Solution A's plant was assessed by investigation of waste streams and the relevant treatments were decided.
\vspace{-10pt}
\subsubsection{Choice of Calcium Loop in Solvay route}
\vspace{-10pt}
As referenced in Section XX in the \textbf{Synthesis Report}, various routes were proposed for the production of soda ash. It was decided that inserting a calcium loop was the best option for safety and environmental reasons. The traditional Solvay process uses ammonia gas to form basic ammoniated brine. Ammonia is a hazardous substance that has many negative impacts for the environment and health of humans. Additionally, ammonia gas is corrosive, may be fatal if inhaled and has a risk explosion, therefore many regulations and precautions have to be taken to store it \citep{storage ammonia}. 
\end{landscape}

\noindent By avoiding its use and introducing CaO to form basic calcium hydroxide one of the main hazards of the Solvay is being prevented. One of the major benefits of calcium looping is that it does not introduce any new species that were not already present in the process. Moreover, Solution A Ltd. by coming up with this novel technique, is applying principles 3 (Less hazardous chemical synthesis) and 12 (Inherently safer chemistry for accident prevention) from the 12 principles of green chemistry (see Appendix \ref{section:12principles}). 
\vspace{-10pt}
\subsubsection{Choice of Operating Pressures and Temperatures}
\vspace{-10pt}
In the Solvay process, a temperature of 1000\textdegree C must be reached for the calcination of limestone because it is a thermal decomposition. Since this temperature is crucial for the reaction, hazards are mitigated through introducing an insulating medium surrounding the outer fuel vessel.A key endeavor of the development of novel processes is the alleviation of hazards in and surrounding the plant. Thus, all process units within Solution A\textquotesingle s plant aim to operate at close to atmospheric pressure (with the exception of the carbonator). Operating at higher pressures increases hazard risks from potential chemical leaks, resulting in a higher F\&EI score. The detailed process unit hazards table with reference to the Dow\textquotesingle s Fire and Explosion Index is seen in Appendix \ref{dowsappendix}. Additionally,it´s necessary to place a pressure relief system in the carbonator to avoid overpressurisation, as well as ensure the safety of surrounding personnel.
\vspace{-10pt}
\subsubsection{Choice of using DSR for R-1}
\vspace{-10pt}
The Direct Separation Reactor was a novel requirement for the development of the plant. Calcination of CaCO$_3$ is performed by indirect heating - as opposed to the traditional kiln - whereby gaseous fuel emissions are theoretically made up of only pure CO$_2$ \citep{LEILAC}. This enables CO$_2$ to be captured without an additional energy-intensive step, and thus minimising the plant's carbon footprint. 
Both fluidised beds and unabated kiln were considered for the decomposition reaction, however compared to the DSR, the kiln lacked in production of clean emissions and the conversion for the fluidised bed was too small for the process at a suitable reactor size. 
\vspace{-10pt}
\subsection{Inherent Safety Considerations}
\vspace{-10pt}
The concept of inherent safety involves the elimination and if not, the reduction of hazards. Inherent safety can be achieved through four main methods which are discussed further in the following sections. 
\vspace{-10pt}
\subsubsection{Minimise}
\vspace{-10pt}
The minimisation of usage of hazardous materials, is a way of achieving inherent safety. In the Solvay process many of the streams are recycled thus minimizing the constant introduction of hazardous materials in the process. For example, the stream of CO$_2$ is reused to bubble the calcium hydroxide in the bubble column, the CaO flow is re-utilized to form Ca(OH)$_2$ and the CaCO$_3$ stream is recycled as a raw material therefore a constant supply of it is not needed which is also economically beneficial.  
\vspace{-10pt}
\subsubsection{Substitute}
\vspace{-10pt}
Substitution is another method to make the process inherently safer by replacing hazardous substances with safer alternatives. This principle was applied when choosing the synthesis route. As previously mentioned the traditional Solvay process uses ammonia gas a very hazardous substance from an environmental, storing and safety aspect. By replacing it with calcium oxide a much safer option, the process becomes less dangerous. Moreover, the unabated kiln was substituted by a direct separator reactor. When a traditional lime kiln is used, the concentration of CO$_2$ present in the fuel gas obtained is not enough for effective sequestration and will introduce impurities into the rest of the process. On the other hand, when using a DSR, a stream of pure CO$_2$ is attained and ready to be sequestered. 
\vspace{-10pt}
\subsubsection{Moderate}
\vspace{-10pt}
Moderating is achieved by the reduction of the consequence severity of a particular hazard. Examples of moderation within Solution A's plant include ensuring that when possible, reactions and separations operate under standard conditions. However, since the decomposition of CaCO$_3$ relies on operating temperatures greater 900\textdegrees C. Furthermore, the plant will be structured in a way such that process equipment is spaced apart in order to reduce the probability of the process units from causing damage to other units and/or people in the case of an explosion.
\subsubsection{Simplify}
\vspace{-10pt}
The last method of considering inherent safety is simplification of the overall soda ash production process. The site layout was simplified by using the Principle of Flow, which optimises the efficiency of production in the physical sense \citep{flow}. In this case, the pipes between each process unit was minimised and never interfere with each other. In terms of the process route, the species required to form Na$_2$CO$_3$ was reduced so as not to include any unnecessary reagents/catalysts in the reactors.  
\subsection{Active Safety Equipment}
\vspace{-10pt}
Even though the four methods of inherent safety are very effective in the minimisation of risks, there are some hazards that cannot be eliminated or reduced. For these situations the implementation of safety equipment is required in the plant process. 

\noindent In the case of the DSR and carbonator, vessel temperatures will be extremely high and so precaution equipment must be put in place to protect the surrounding environment. 

%Find potential cases in the flow diagram where e.g. an explosion can occur due to overpressurisation 
% I think for some of this section we need to consult the control group!