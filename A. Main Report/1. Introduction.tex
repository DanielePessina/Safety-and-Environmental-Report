\section{Safety \& Environmental Vision}
\vspace{-10pt}
\noindent Soda ash is widely utilised as a feedstock material in many industries. The conventional product pathway is the Solvay process, consisting in the reaction of CO$_2$ with sodium-rich brine and limestone inducting an ammonia catalyst. The main disadvantages associated with this pathway are the large carbon footprint and ammonia material, which is not only a hazardous chemical for the environment but a notable hazard to humans \citep{Safetydatasheet}. Aligning with EU decarbonisation goals \citep{HouseoflordsEU}, Solution A industries has designed a novel, environmentally-conscious soda ash process. Fuel emissions of typical soda ash plants are 500kg of CO$_2$ released to the atmosphere per tonne of product \citep{parisagreement}. Through the implementation of the direct separation reactor (DSR), Solution A Ltd. has taken the initiative to produce less than 30kg per tonne of Na$_2$CO$_3$ at a scale of 1Mtonne of product per annum, vastly decreasing carbon emissions, and introducing a calcium loop to avoid the use of ammonia. The preliminary process design and business feasibility analysis are enclosed in this report \citep{osmosis}. 







